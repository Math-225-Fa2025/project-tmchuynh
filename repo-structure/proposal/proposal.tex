% Options for packages loaded elsewhere
\PassOptionsToPackage{unicode}{hyperref}
\PassOptionsToPackage{hyphens}{url}
\documentclass[
]{article}
\usepackage{xcolor}
\usepackage[margin=1in]{geometry}
\usepackage{amsmath,amssymb}
\setcounter{secnumdepth}{5}
\usepackage{iftex}
\ifPDFTeX
  \usepackage[T1]{fontenc}
  \usepackage[utf8]{inputenc}
  \usepackage{textcomp} % provide euro and other symbols
\else % if luatex or xetex
  \usepackage{unicode-math} % this also loads fontspec
  \defaultfontfeatures{Scale=MatchLowercase}
  \defaultfontfeatures[\rmfamily]{Ligatures=TeX,Scale=1}
\fi
\usepackage{lmodern}
\ifPDFTeX\else
  % xetex/luatex font selection
\fi
% Use upquote if available, for straight quotes in verbatim environments
\IfFileExists{upquote.sty}{\usepackage{upquote}}{}
\IfFileExists{microtype.sty}{% use microtype if available
  \usepackage[]{microtype}
  \UseMicrotypeSet[protrusion]{basicmath} % disable protrusion for tt fonts
}{}
\makeatletter
\@ifundefined{KOMAClassName}{% if non-KOMA class
  \IfFileExists{parskip.sty}{%
    \usepackage{parskip}
  }{% else
    \setlength{\parindent}{0pt}
    \setlength{\parskip}{6pt plus 2pt minus 1pt}}
}{% if KOMA class
  \KOMAoptions{parskip=half}}
\makeatother
\usepackage{color}
\usepackage{fancyvrb}
\newcommand{\VerbBar}{|}
\newcommand{\VERB}{\Verb[commandchars=\\\{\}]}
\DefineVerbatimEnvironment{Highlighting}{Verbatim}{commandchars=\\\{\}}
% Add ',fontsize=\small' for more characters per line
\usepackage{framed}
\definecolor{shadecolor}{RGB}{248,248,248}
\newenvironment{Shaded}{\begin{snugshade}}{\end{snugshade}}
\newcommand{\AlertTok}[1]{\textcolor[rgb]{0.94,0.16,0.16}{#1}}
\newcommand{\AnnotationTok}[1]{\textcolor[rgb]{0.56,0.35,0.01}{\textbf{\textit{#1}}}}
\newcommand{\AttributeTok}[1]{\textcolor[rgb]{0.13,0.29,0.53}{#1}}
\newcommand{\BaseNTok}[1]{\textcolor[rgb]{0.00,0.00,0.81}{#1}}
\newcommand{\BuiltInTok}[1]{#1}
\newcommand{\CharTok}[1]{\textcolor[rgb]{0.31,0.60,0.02}{#1}}
\newcommand{\CommentTok}[1]{\textcolor[rgb]{0.56,0.35,0.01}{\textit{#1}}}
\newcommand{\CommentVarTok}[1]{\textcolor[rgb]{0.56,0.35,0.01}{\textbf{\textit{#1}}}}
\newcommand{\ConstantTok}[1]{\textcolor[rgb]{0.56,0.35,0.01}{#1}}
\newcommand{\ControlFlowTok}[1]{\textcolor[rgb]{0.13,0.29,0.53}{\textbf{#1}}}
\newcommand{\DataTypeTok}[1]{\textcolor[rgb]{0.13,0.29,0.53}{#1}}
\newcommand{\DecValTok}[1]{\textcolor[rgb]{0.00,0.00,0.81}{#1}}
\newcommand{\DocumentationTok}[1]{\textcolor[rgb]{0.56,0.35,0.01}{\textbf{\textit{#1}}}}
\newcommand{\ErrorTok}[1]{\textcolor[rgb]{0.64,0.00,0.00}{\textbf{#1}}}
\newcommand{\ExtensionTok}[1]{#1}
\newcommand{\FloatTok}[1]{\textcolor[rgb]{0.00,0.00,0.81}{#1}}
\newcommand{\FunctionTok}[1]{\textcolor[rgb]{0.13,0.29,0.53}{\textbf{#1}}}
\newcommand{\ImportTok}[1]{#1}
\newcommand{\InformationTok}[1]{\textcolor[rgb]{0.56,0.35,0.01}{\textbf{\textit{#1}}}}
\newcommand{\KeywordTok}[1]{\textcolor[rgb]{0.13,0.29,0.53}{\textbf{#1}}}
\newcommand{\NormalTok}[1]{#1}
\newcommand{\OperatorTok}[1]{\textcolor[rgb]{0.81,0.36,0.00}{\textbf{#1}}}
\newcommand{\OtherTok}[1]{\textcolor[rgb]{0.56,0.35,0.01}{#1}}
\newcommand{\PreprocessorTok}[1]{\textcolor[rgb]{0.56,0.35,0.01}{\textit{#1}}}
\newcommand{\RegionMarkerTok}[1]{#1}
\newcommand{\SpecialCharTok}[1]{\textcolor[rgb]{0.81,0.36,0.00}{\textbf{#1}}}
\newcommand{\SpecialStringTok}[1]{\textcolor[rgb]{0.31,0.60,0.02}{#1}}
\newcommand{\StringTok}[1]{\textcolor[rgb]{0.31,0.60,0.02}{#1}}
\newcommand{\VariableTok}[1]{\textcolor[rgb]{0.00,0.00,0.00}{#1}}
\newcommand{\VerbatimStringTok}[1]{\textcolor[rgb]{0.31,0.60,0.02}{#1}}
\newcommand{\WarningTok}[1]{\textcolor[rgb]{0.56,0.35,0.01}{\textbf{\textit{#1}}}}
\usepackage{graphicx}
\makeatletter
\newsavebox\pandoc@box
\newcommand*\pandocbounded[1]{% scales image to fit in text height/width
  \sbox\pandoc@box{#1}%
  \Gscale@div\@tempa{\textheight}{\dimexpr\ht\pandoc@box+\dp\pandoc@box\relax}%
  \Gscale@div\@tempb{\linewidth}{\wd\pandoc@box}%
  \ifdim\@tempb\p@<\@tempa\p@\let\@tempa\@tempb\fi% select the smaller of both
  \ifdim\@tempa\p@<\p@\scalebox{\@tempa}{\usebox\pandoc@box}%
  \else\usebox{\pandoc@box}%
  \fi%
}
% Set default figure placement to htbp
\def\fps@figure{htbp}
\makeatother
\setlength{\emergencystretch}{3em} % prevent overfull lines
\providecommand{\tightlist}{%
  \setlength{\itemsep}{0pt}\setlength{\parskip}{0pt}}
\usepackage[]{natbib}
\bibliographystyle{plainnat}
\usepackage{bookmark}
\IfFileExists{xurl.sty}{\usepackage{xurl}}{} % add URL line breaks if available
\urlstyle{same}
\hypersetup{
  pdftitle={Project proposal},
  pdfauthor={Tina Huynh},
  hidelinks,
  pdfcreator={LaTeX via pandoc}}

\title{Project proposal}
\author{Tina Huynh}
\date{}

\begin{document}
\maketitle

{
\setcounter{tocdepth}{3}
\tableofcontents
}
\section{Executive Summary}\label{executive-summary}

This project examines the illegal trade in tiger parts and its
consequences for biodiversity, human communities, and public health. We
will quantify seizure trends and trade signals, place them in the
context of wild versus ex-situ (zoo/facility) populations, and integrate
socioeconomic and governance indicators that shape poaching incentives
and enforcement capacity. We will also address the historic loss of
tiger subspecies/populations and the risks of zoonotic spillover
associated with unregulated live-animal trade.

Results will be presented as a reproducible data analysis (R/Quarto), a
compact dataset, and a short report linking empirical findings to
policy-relevant implications: conservation effectiveness, community
livelihoods, and public-health prevention.

\subsection{Scope}\label{scope}

\begin{itemize}
\item
  \textbf{Focal taxon}: Panthera tigris (tigers).
\item
  \textbf{Spatial unit}: country (aggregate to sub-regions where
  needed).
\item
  \textbf{Temporal scope}: 2010--2024 (or longest consistent overlap
  across sources).
\end{itemize}

\subsection{Objective}\label{objective}

\begin{itemize}
\item
  \textbf{Trade/Enforcement}: Quantify and visualize tiger-part seizure
  trends over time and geography; compare item types (skins, bones,
  teeth/claws).
\item
  \textbf{Prices (carefully handled)}: Use vetted wholesale price
  ranges/indices to explore whether market signals correlate with
  recorded pressure (seizures).
\item
  \textbf{Populations}: Contextualize trends with wild population
  estimates and, where access permits, ex-situ counts (snapshot).
\item
  \textbf{Extinctions}: Document historic subspecies/population
  extinctions and recent country-level functional extirpations,
  acknowledging multiple interacting drivers.
\item
  \textbf{Socioeconomics \& Governance}: Test associations between
  poverty, governance, and recorded poaching pressure; articulate the
  dependency cycle of illegal income.
\item
  \textbf{Public Health}: Highlight how unregulated live-animal trade
  elevates spillover risk and burdens human health systems.
\end{itemize}

\section{Background and Problem
Statement}\label{background-and-problem-statement}

\subsection{The Problem}\label{the-problem}

Illegal trade in tiger parts (bones, skins, teeth/claws) and live
animals continues to exert pressure on already small, fragmented wild
populations. Poaching is reinforced by poverty, weak governance, and
organized-crime supply chains; in some landscapes, snaring for
commercial bushmeat and parts has become pervasive. Beyond biodiversity
loss, this crime undermines the rule of law, corrodes public
institutions, and erodes nature-based tourism revenues and jobs that
would otherwise finance conservation and local livelihoods.

Segments of the trade move live wildlife through multi-species, crowded,
and unsanitary holding/transport. These conditions heighten the risk of
zoonotic spillover by increasing close contact and pathogen shedding,
and they bypass veterinary and quarantine controls---creating costs for
public health as well as conservation. The risk is well documented
across intergovernmental guidance and reviews and is relevant to any
policy discussion on wildlife crime.

The economic stakes are material. Tiger-viewing tourism can generate
substantial site-level revenue and local employment; poaching-driven
declines in sightings or access jeopardize those flows and can push
households back toward illegal harvest, reinforcing a negative
income--poaching loop. Countries that do not control poaching also incur
reputational harms that affect destination branding, investor
confidence, and international cooperation.

\subsection{\texorpdfstring{The History of the \emph{Panthera tigris}
Population}{The History of the Panthera tigris Population}}\label{the-history-of-the-panthera-tigris-population}

\textbf{From continental giant to fragmented remnant.}\\
Historically, tigers ranged from the forests and river systems of
Anatolia/Central Asia through the Indian subcontinent and Southeast Asia
to the Russian Far East and the Sunda Islands. By the early 1900s they
occupied a mosaic of habitats---tropical moist forests, dry deciduous
forests, mangroves, temperate/boreal forests, and grass--shrub
systems---supported by abundant wild ungulates.

\textbf{1900--1970: Rapid contraction and three extinctions.}\\
- \textbf{Persecution \& sport hunting:} Colonial/royal hunting, bounty
programs, and better firearms drove sustained offtake of tigers and
their prey.\\
- \textbf{Land conversion:} Large-scale clearing for agriculture,
timber, plantations, and transport infrastructure fragmented habitat and
severed corridors.\\
- \textbf{Outcome:} Island subspecies \textbf{Bali} and \textbf{Javan}
tigers, and the \textbf{Caspian} tiger on the mainland, disappeared
during the mid-to-late 20th century as small, isolated populations
collapsed under combined hunting pressure, prey loss, and habitat
conversion.

\textbf{1970s--1990s: International controls and uneven responses.}\\
- \textbf{Policy era:} The 1970s--80s brought protected-area expansion
and international trade controls; India launched \textbf{Project Tiger
(1973)}, which created a dedicated reserve network and specialized
protection.\\
- \textbf{Shifting demand:} Illegal trade in \textbf{skins} and
\textbf{bones} (for luxury and traditional medicine uses) escalated;
some countries strengthened enforcement while others struggled with weak
institutions and porous borders.\\
- \textbf{Regional contrasts:} The \textbf{Amur (Siberian)} tiger
hovered near the brink but began recovering under stricter protection;
parts of mainland Southeast Asia saw steep declines as snaring
proliferated.

\textbf{2000s: The ``snaring crisis'' and national extirpations.}\\
- \textbf{Low-cost mass killing:} Industrial-scale deployment of cheap
wire snares by commercial and subsistence actors removed both tigers and
their prey.\\
- \textbf{National losses:} Tigers vanished from \textbf{Cambodia}
(declared functionally extinct later, with reintroduction now planned)
and from key landscapes in \textbf{Laos} and \textbf{Viet Nam}.
\textbf{Malaysia} began a sharp decline despite substantial forest
cover, while \textbf{Indonesia's} \textbf{Sumatran} tiger persisted
under high pressure.\\
- \textbf{Bright spots:} \textbf{India} and \textbf{Nepal} expanded
camera-trap monitoring, boosted protection, and saw gradual increases;
\textbf{Bhutan} remained relatively stable; the \textbf{Russian Far
East} consolidated gains.

\textbf{2010s--2020s: TX2 ambitions, taxonomy update, cautious
optimism.}\\
- \textbf{TX2 commitment (2010):} Range states endorsed the goal to
\textbf{double wild tiger numbers by 2022}, catalyzing funding,
monitoring, and anti-poaching partnerships.\\
- \textbf{Taxonomy (2017):} Formal revision consolidated tigers into two
subspecies---\textbf{continental} (\emph{P. t. tigris}) and
\textbf{Sunda} (\emph{P. t. sondaica})---while retaining the
conservation significance of historically recognized forms (e.g., Bali,
Javan, Caspian).\\
- \textbf{Status (recent):} The latest global assessment reports
\textbf{\textasciitilde3.7k--5.6k} wild tigers. Gains in \textbf{India,
Nepal, Bhutan, Russia} and re-occupancy into parts of \textbf{northeast
China} contrast with ongoing declines or very low numbers in portions of
mainland Southeast Asia. The \textbf{South China} tiger remains
\textbf{likely extinct in the wild}.

\textbf{Structure today: small, isolated, corridor-dependent
populations.}\\
Most surviving subpopulations are small and geographically isolated,
connected---if at all---by narrow, human-dominated corridors. Priority
landscapes (e.g., \textbf{Terai Arc}, \textbf{Western Ghats--Central
India}, \textbf{Sundarbans}, \textbf{Russian Far East--NE China})
function as meta-populations where connectivity, prey recovery, and
conflict mitigation determine long-term viability. Fragmentation raises
risks of \textbf{genetic erosion}, \textbf{demographic stochasticity},
and site-level extirpation.

\textbf{Threat matrix (persistent and interacting):}\\
- \textbf{Illegal killing \& trade:} Targeted poaching for \textbf{skins
and bones}, opportunistic killing via \textbf{snares}, and retaliatory
conflict killings.\\
- \textbf{Prey depletion:} Overhunting of deer, wild pig, and other
ungulates undermines carrying capacity even where habitat remains.\\
- \textbf{Habitat loss \& fragmentation:} Agricultural expansion,
plantation estates, logging, linear infrastructure, and settlements
sever movement pathways.\\
- \textbf{Governance \& poverty:} Weak enforcement capacity, corruption,
and limited rural livelihoods fuel participation in illegal supply
chains.\\
- \textbf{Emerging risks:} \textbf{Climate change} threatens
coastal/mangrove habitats (e.g., \textbf{Sundarbans} via sea-level rise
and cyclones); expanding road/rail networks elevate mortality and open
access for poachers.

\textbf{What has worked (evidence-backed levers):}\\
- \textbf{Site protection \& patrol quality:} Skilled, well-resourced
patrols (SMART, informant networks), rapid response, and judicial
follow-through reduce poaching.\\
- \textbf{Prey base recovery:} Community co-management and hunting
controls that allow ungulates to rebound increase tiger carrying
capacity.\\
- \textbf{Corridor protection:} Securing key linkages, regulating linear
infrastructure, and targeted restoration maintain meta-population
function.\\
- \textbf{Community benefits:} Genuine local income from
\textbf{ecotourism}, conservation jobs, and performance-based payments
improves tolerance and reduces incentives to poach.\\
- \textbf{Demand reduction \& market controls:} Visible enforcement,
cross-border cooperation, and sustained demand-reduction campaigns
shrink margins for traffickers.\\
- \textbf{Transboundary collaboration:} Joint monitoring and coordinated
enforcement in shared landscapes (e.g., India--Nepal Terai,
Russia--China Far East) address cross-border movement of wildlife and
offenders.

\textbf{Bottom line.}\\
Tigers can rebound rapidly where \textbf{protection is credible},
\textbf{prey is abundant}, and \textbf{communities see real benefits}.
But the same factors that drove the 20th-century collapse---illegal
killing, prey loss, and habitat fragmentation---still operate. Without
continuous, well-governed effort that pairs \textbf{livelihoods and
governance} with \textbf{enforcement and demand reduction}, local
recoveries remain fragile and reversible.

\subsection{Existing Research}\label{existing-research}

Wildlife crime, governance, and impacts. The UNODC World Wildlife Crime
Report (2024) documents the scale of wildlife trafficking, its
convergence with corruption and organized crime, and harms to
governance, development, and public health---framing wildlife crime as a
cross-cutting development and security issue, not just a conservation
problem.

Poverty, livelihoods, and poaching. Reviews in conservation and
development literatures show that illegal hunting is often rooted in
poverty, limited legal opportunities, and weak state presence; snaring
in Southeast Asia is repeatedly highlighted as a low-cost, high-impact
method linked to commercial demand and household income needs. The
evidence cautions against simplistic narratives and emphasizes
structural drivers (poverty, markets, governance).

Population status and trends. The IUCN Red List reassessment (Goodrich
et al., 2022) provides the current global estimate and Endangered
status; WWF/Global Tiger Forum syntheses note stabilization or increases
in some range states since 2010 (TX2), alongside declines
elsewhere---indicating heterogeneous outcomes linked to protection and
pressure.

Snaring and recent extirpations. Peer-reviewed field studies from Laos
(Nam Et-Phou Louey) document the disappearance of tigers by 2014, with
snares implicated; WWF and news summaries align with those findings.
These cases are used as cautionary examples of how trade-driven killing,
coupled with habitat and prey loss, can rapidly eliminate small
populations.

Health risk from live trade. Intergovernmental guidelines (WOAH) and
UNODC communication during COVID-19 underline that unregulated
live-animal trade elevates spillover risk through high-contact,
unsanitary interfaces across the supply chain. This evidence underpins
the project's public-health framing (associational, not causal claims).

Tourism, jobs, and finance. WWF and site-level studies report
substantial tiger-linked tourism value and local capture of benefits;
declines in wildlife or access translate into measurable revenue and
employment losses, weakening incentives for protection and local buy-in.
These findings justify including an ecotourism/jobs lens alongside
biological and enforcement metrics.

\section{Research Questions}\label{research-questions}

\subsection{Core Research Questions}\label{core-research-questions}

\begin{itemize}
\item
  \textbf{Q:} Where wild population snapshots exist, is wild\_pop\_mid
  negatively correlated with seizures per million (signaling depletion
  pressure) or uncorrelated (detection bias)?\\
  \textbf{H:} Weak negative correlation overall, with wide uncertainty.
\item
  \textbf{Q:} Do persistent poaching signals plausibly damage
  destination brand and cooperation prospects?\\
  \textbf{H:} High, sustained illegal trade indicators correlate with
  reputational risk narratives.
\item
  \textbf{Q:} What economic losses (revenue/jobs) are implied when tiger
  sightings/access fall (e.g., from poaching)?\\
  \textbf{H:} Reduced sightings/access materially cuts reserve-level
  revenue and local earnings; at scale, this undermines sustainable
  employment, reinforcing the poverty--poaching loop.
\item
  \textbf{Q:} Do improvements (Δ) in governance indices correlate with
  subsequent decreases in seizures per million (lead--lag association)?
  \textbf{H:} Positive Δ rule-of-law in year t associates with lower
  seizures per million in t+1.
\item
  \textbf{Q:} Are poverty, unemployment, basic services, and governance
  associated with recorded poaching pressure?\\
  \textbf{H:} Higher poverty and weaker rule-of-law align with higher
  seizures per capita.
\item
  \textbf{Q:} Are year-over-year changes in poverty/unemployment
  correlated with changes in seizures per million (Δ--Δ analysis)?\\
  \textbf{H:} Worsening poverty/unemployment associates with rising
  seizures per million.
\item
  \textbf{Q:} How do trade/enforcement trends relate to wild population
  context and (if available) ex-situ counts?\\
  \textbf{H:} Areas with historically depleted wild populations show
  weaker elasticity of seizures to price signals (fewer animals to
  poach), but trafficking routes persist.
\end{itemize}

\subsection{Sub-Questions}\label{sub-questions}

\begin{itemize}
\item
  \textbf{Q:} How have tiger-part seizures changed over time and where
  are hot spots?\\
  \textbf{H:} Seizures show clustered hot spots and non-linear trends
  driven by enforcement and market dynamics.
\item
  \textbf{Q:} Do wholesale price indices/ranges align with subsequent
  changes in seizures?\\
  \textbf{H:} Higher prior-year wholesale indices correlate with higher
  current seizures (association, not causal proof).
\item
  \textbf{Q:} What does the subspecies/population extinction timeline
  indicate about illegal trade alongside habitat/prey loss?\\
  \textbf{H:} Extirpation events coincide with periods of high poaching
  pressure and weak governance, with habitat/prey loss as interacting
  drivers.
\item
  \textbf{Q:} Is the poverty ↔ seizures correlation heterogeneous by
  region (South Asia vs.~mainland Southeast Asia vs.~Russia/China)?\\
  \textbf{H:} Strongest in mainland Southeast Asia, weaker in
  higher-income/high-governance regions.
\item
  \textbf{Q:} Where live-animal seizures occur, do indicators suggest
  elevated spillover risk?\\
  \textbf{H:} Higher shares of live-trade seizures co-occur with weaker
  services/governance.
\end{itemize}

\section{Data}\label{data}

\subsection{Data Sources}\label{data-sources}

\begin{itemize}
\item
  \textbf{CITES Trade Database} --- official, country-reported
  international trade in CITES-listed species; filter for Panthera
  tigris and relevant product terms.
\item
  \textbf{TRAFFIC (e.g., ``Skin and Bones'' series)} --- consolidated
  seizure analyses (counts/weights by year/country/item).
\item
  \textbf{Protected areas / habitat context} --- WDPA; forest-loss
  layers (Global Forest Watch/Hansen).
\item
  \textbf{IUCN Red List (Panthera tigris)} --- status, range,
  best-available global population context.
\item
  \textbf{UNODC World Wildlife Crime Reports} --- trafficking flows,
  valuation methods, wholesale price ranges/indices (use only as
  ranges/indices; no granular detail).
\item
  \textbf{Wildlife Justice Commission (WJC)} --- carefully collected
  wholesale price intelligence (use as indices/ranges, aggregated by
  year/region).
\item
  \textbf{Species360 / ZIMS \& studbooks} --- ex-situ snapshots
  (access-controlled; use if permission is granted).
\item
  \textbf{Socioeconomics/Governance} --- World Bank (WDI), Worldwide
  Governance Indicators (WGI).
\item
  \textbf{Public-health} --- aggregated country-year counts from
  official reporting where available (used illustratively, not for
  causal claims).
\end{itemize}

\subsection{Variables \& Data Schema}\label{variables-data-schema}

\subsubsection{TRAFFIC --- Seizures (pressure
signal)}\label{traffic-seizures-pressure-signal}

\begin{itemize}
\tightlist
\item
  \textbf{Grain:} \texttt{country}--\texttt{year} (optionally
  \texttt{item})
\item
  \textbf{Vars (raw/clean):} \texttt{item}
  (skin\textbar bone\textbar tooth\textbar claw\textbar mixed),
  \texttt{seizures\_n}, \texttt{seizures\_kg}
\item
  \textbf{Derived:} \texttt{seizures\_total}, \texttt{skins},
  \texttt{bones}, \texttt{teeth\_claws},
  \texttt{seizures\_per\_million\ =\ 1e6\ *\ seizures\_total\ /\ pop},
  \texttt{log\_spm\ =\ log1p(seizures\_per\_million)}
\item
  \textbf{Live-trade proxy:} \texttt{live\_flag} → \texttt{live\_share}
  (\% by country--year)
\end{itemize}

\subsubsection{CITES Trade Database --- Reported trade
(context)}\label{cites-trade-database-reported-trade-context}

\begin{itemize}
\tightlist
\item
  \textbf{Grain:} record → aggregate to \texttt{country}--\texttt{year}
\item
  \textbf{Variables:} \texttt{exporter}, \texttt{importer},
  \texttt{year}, \texttt{term}, \texttt{purpose}, \texttt{source},
  \texttt{reported\_qty}, \texttt{unit}
\item
  \textbf{Derived:} \texttt{cites\_records}, \texttt{cites\_qty\_total},
  \texttt{cites\_live\_share}
\end{itemize}

\subsubsection{\texorpdfstring{UNODC / WJC --- \textbf{Wholesale price
indices/ranges}}{UNODC / WJC --- Wholesale price indices/ranges}}\label{unodc-wjc-wholesale-price-indicesranges}

\begin{itemize}
\tightlist
\item
  \textbf{Grain:} \texttt{region}--\texttt{year}
\item
  \textbf{Variables:} \texttt{commodity},
  \texttt{market\_stage="wholesale"}, \texttt{index\_low\_usd},
  \texttt{index\_high\_usd}, \texttt{index\_mid},
  \texttt{index\_mid\_lag}
\item
  \textbf{Note:} Optional CPI-deflated \texttt{index\_mid\_real}
\end{itemize}

\subsubsection{IUCN / National sources --- Population
context}\label{iucn-national-sources-population-context}

\begin{itemize}
\tightlist
\item
  \textbf{Wild pop (snapshot):}
  \texttt{wild\_pop\_min\textbar{}max\textbar{}mid},
  \texttt{estimate\_year} (country/landscape)
\item
  \textbf{Extinction / extirpation events:} \texttt{geography},
  \texttt{event\_type}, \texttt{event\_year}, \texttt{notes}
\end{itemize}

\subsubsection{Species360 / ZIMS / Studbooks --- Ex-situ
(snapshot)}\label{species360-zims-studbooks-ex-situ-snapshot}

\begin{itemize}
\tightlist
\item
  \textbf{Variables:} \texttt{ex\_situ\_count},
  \texttt{facility\_count}, \texttt{ex\_situ\_year},
  \texttt{coverage\_notes}
\end{itemize}

\subsubsection{WDI \& WGI --- Socioeconomics \&
governance}\label{wdi-wgi-socioeconomics-governance}

\begin{itemize}
\tightlist
\item
  \textbf{Demographics \& income:} \texttt{pop}, \texttt{gdp\_ppp\_pc}
\item
  \textbf{Poverty \& employment:} \texttt{pov\_215}, \texttt{unemp}
\item
  \textbf{Basic services:} \texttt{clean\_water} or
  \texttt{electricity\_access}
\item
  \textbf{Governance:} \texttt{rule\_of\_law},
  \texttt{control\_of\_corruption}
\item
  \textbf{Plus:} \texttt{region} (for joining to price indices)
\end{itemize}

\subsubsection{WDPA / GFW}\label{wdpa-gfw}

\begin{itemize}
\tightlist
\item
  \textbf{Protection coverage:} \texttt{pa\_coverage\_pct}
\item
  \textbf{Habitat loss:} \texttt{forest\_loss\_pct},
  \texttt{baseline\_forest\_cover}
\end{itemize}

\subsubsection{Public-health proxies}\label{public-health-proxies}

\begin{itemize}
\tightlist
\item
  \textbf{Live trade:} \texttt{live\_share}
\item
  \textbf{Health signal:} \texttt{outbreak\_count} (coarse)
\end{itemize}

\subsubsection{Tourism anchors
(site-level)}\label{tourism-anchors-site-level}

\begin{itemize}
\tightlist
\item
  \textbf{Variables:} \texttt{reserve}, \texttt{anchor\_value\_usd},
  \texttt{local\_share}, \texttt{gate\_fees\_usd},
  \texttt{reserve\_budget\_usd}, \texttt{year}
\item
  \textbf{Scenarios:} \texttt{visitation\_decline\_\{10,20,30\}},
  \texttt{revenue\_loss}, \texttt{local\_earnings\_loss}
\end{itemize}

\subsubsection{Keys \& Joins}\label{keys-joins}

\begin{itemize}
\tightlist
\item
  \textbf{Primary panel key:} \texttt{country} (ISO3) + \texttt{year}
\item
  \textbf{Regional joins:} map \texttt{country\ →\ region} for price
  indices
\item
  \textbf{Snapshots:} keep separate with their own year columns; do
  \textbf{not} annualize
\end{itemize}

\section{Methodology}\label{methodology}

\subsection{Data Analysis Plan}\label{data-analysis-plan}

\subsubsection{Data Quality \& Bias
Controls}\label{data-quality-bias-controls}

\begin{itemize}
\tightlist
\item
  Harmonize ISO3 country codes and region mapping; explicit missingness
  handling.
\item
  Use per-capita and log transforms for skewed outcomes (seizures).
\item
  Seizures are detection-biased; population estimates are
  ranges/snapshots.
\item
  Price data used \textbf{only} as aggregated \textbf{wholesale}
  ranges/indices.
\end{itemize}

\subsubsection{Prices → Pressure
(associations)}\label{prices-pressure-associations}

\begin{itemize}
\tightlist
\item
  \textbf{Unit:} region--year (or country aggregated to region).
\item
  \textbf{Spec:}
  \(\text{spm}_t = \alpha + \beta\,\text{index\_mid}_{t-1} + \gamma_t + \varepsilon\)
  (OLS on SPM; GLM on counts with offset log(pop)).
\item
  Readout: sign/CI of β; LOESS overlay. Associations only.
\end{itemize}

\subsubsection{Populations \&
Extinctions}\label{populations-extinctions}

\begin{itemize}
\tightlist
\item
  Timeline of historic subspecies/population losses and recent country
  extirpations; overlay regional seizure trends.
\item
  Place seizure trends alongside wild population context; interpret
  descriptively.
\end{itemize}

\subsubsection{Socioeconomic Drivers \& Dependency
Cycle}\label{socioeconomic-drivers-dependency-cycle}

\begin{itemize}
\tightlist
\item
  \textbf{Unit:} country--year.
\item
  Bivariate plots: poverty, unemployment, basic services, governance vs
  \(\log(1+\text{SPM})\).
\item
  Light regression with lagged covariates and year FE:
  \(\text{Seizures}_{c,t} = \beta_0 + \beta_1 \text{Poverty}_{c,t-1} + \beta_2 \text{RuleOfLaw}_{c,t-1} + \beta_3 \text{GDPpc}_{c,t-1} + \gamma_t + \varepsilon_{c,t}\)
\end{itemize}

\subsubsection{Live-Trade \& Public-Health
Risk}\label{live-trade-public-health-risk}

\begin{itemize}
\tightlist
\item
  \textbf{Unit:} country--year (subset with \texttt{live\_share}).
\item
  \textbf{Spec:}
  \texttt{live\_share\ \textasciitilde{}\ services\ +\ governance\ +\ year\ FE}
  (OLS or beta regression).
\item
  Use panels to illustrate elevated spillover risk where live-trade
  share is high.
\end{itemize}

\subsubsection{Ecotourism Revenue \&
Jobs}\label{ecotourism-revenue-jobs}

\begin{itemize}
\tightlist
\item
  Site-level anchor(s); scenario analysis (10--30\% visitation drop) →
  revenue and local-earnings losses.
\item
  Contextualize with macro wildlife-tourism jobs figures (qualitative).
\end{itemize}

\subsection{Statistical Tests}\label{statistical-tests}

\textbf{Associations \& correlation}

\begin{itemize}
\tightlist
\item
  Pearson/Spearman between \texttt{log\_spm} and predictors
  (\texttt{pov\_215}, \texttt{gdp\_ppp\_pc}, \texttt{rule\_of\_law},
  \texttt{index\_mid\_lag},
  \texttt{clean\_water}/\texttt{electricity\_access}).
\item
  Kendall's τ for small-N regional series.
\end{itemize}

\textbf{Group/comparative}

\begin{itemize}
\tightlist
\item
  t-test or Wilcoxon for high vs low poverty (or governance) on
  \texttt{log\_spm}.
\item
  ANOVA or Kruskal--Wallis across quartiles; Dunn (Holm) post-hoc if
  nonparametric.
\end{itemize}

\textbf{Trend \& structural change}

\begin{itemize}
\tightlist
\item
  Mann--Kendall + Theil--Sen for monotonic trends in yearly seizures.
\item
  Change-points (CUSUM, Bai--Perron) on seizures or price-index series.
\end{itemize}

\textbf{Count-model diagnostics}

\begin{itemize}
\tightlist
\item
  Overdispersion test (Cameron--Trivedi); switch to Negative Binomial if
  needed.
\item
  Vuong/LR to compare Poisson vs NB vs zero-inflated.
\end{itemize}

\textbf{Panel \& regression diagnostics}

\begin{itemize}
\tightlist
\item
  Breusch--Pagan (heteroskedasticity) → HC3 or cluster-robust SE.
\item
  VIF (multicollinearity); Moran's I on residuals; Ramsey RESET;
  residual plots.
\end{itemize}

\textbf{Proportions \& live-trade}

\begin{itemize}
\tightlist
\item
  Two-proportion z-tests or χ² for binarized comparisons; rely on beta
  regression for continuous \texttt{live\_share}.
\end{itemize}

\textbf{Multiple comparisons \& uncertainty}

\begin{itemize}
\tightlist
\item
  FDR control (Benjamini--Hochberg).
\item
  Bootstrap CIs for medians and coefficients where needed.
\end{itemize}

\subsubsection{Algorithms}\label{algorithms}

\textbf{Core models (panel/cross-section)}

\begin{itemize}
\item
  \textbf{OLS with fixed effects}:

  \begin{itemize}
  \tightlist
  \item
    \texttt{log\_spm\_ct\ \textasciitilde{}\ poverty\_\{t-1\}\ +\ rule\_of\_law\_\{t-1\}\ +\ gdp\_ppp\_pc\_\{t-1\}\ +\ year\_FE}
    (country--year panel).
  \item
    Robust \textbf{HC3} or \textbf{clustered SE} (by country or region).
  \end{itemize}
\item
  \textbf{GLM for counts} (when modeling raw seizures):

  \begin{itemize}
  \tightlist
  \item
    \textbf{Poisson / Negative Binomial} with
    \texttt{offset\ =\ log(pop)}; choose \textbf{NB} if overdispersion;
    consider \textbf{ZINB} if many zeros.
  \end{itemize}
\end{itemize}

\textbf{Nonlinearity \& robust effects}

\begin{itemize}
\tightlist
\item
  \textbf{GAM (splines)} for potential curvature in \texttt{poverty},
  \texttt{rule\_of\_law}, or \texttt{index\_mid\_lag}.
\item
  \textbf{Quantile regression} (e.g., τ = 0.5) to estimate median
  associations robust to outliers.
\end{itemize}

\textbf{``Price → pressure'' association (regional)}

\begin{itemize}
\item
  Regional aggregation + \textbf{lagged regressor}:

  \begin{itemize}
  \tightlist
  \item
    \texttt{spm\_region\_t\ \textasciitilde{}\ index\_mid\_\{t-1\}\ +\ year\_FE},
    OLS with HAC (Newey--West) SE for serial correlation.
  \end{itemize}
\item
  \textbf{LOESS} smoother in scatterplots for visualization (no
  inference).
\end{itemize}

\textbf{Event/Policy analyses (descriptive, not causal claims)}

\begin{itemize}
\tightlist
\item
  \textbf{Interrupted time series (ITS)} / segmented regression around
  major policy shocks, with \textbf{Newey--West} errors.
\item
  \textbf{Event-study plots} (two-way FE if sample permits) to visualize
  pre/post patterns; interpret cautiously.
\end{itemize}

\textbf{Live-trade risk modeling}

\begin{itemize}
\tightlist
\item
  \textbf{Beta regression} (or \textbf{quasi-binomial}) for
  \texttt{live\_share\ \textasciitilde{}\ services\ +\ governance\ +\ year}
\end{itemize}

\textbf{Spatial sanity}

\begin{itemize}
\tightlist
\item
  If Moran's I flags spatial autocorrelation, use \textbf{cluster-robust
  SE by region}, or add \textbf{regional FE}; full spatial lag/error
  models are optional stretch goals.
\end{itemize}

\textbf{Predictive/importance (optional, clearly labeled exploratory)}

\begin{itemize}
\tightlist
\item
  \textbf{Regularized regression} (\textbf{LASSO / Elastic Net}) to
  gauge variable importance among many correlated covariates; k-fold CV
  for tuning.
\item
  \textbf{Random Forest / Gradient Boosting} to obtain
  \textbf{permutation importance} and \textbf{partial dependence} plots;
  stress \textbf{non-causal}, exploratory nature.
\end{itemize}

\textbf{Smoothers \& decomposition (EDA)}

\begin{itemize}
\tightlist
\item
  \textbf{STL decomposition} on longer regional series (seizures or
  price indices) to separate trend/seasonal/irregular components (if
  periodicity exists).
\item
  \textbf{Theil--Sen} lines on bivariate plots as robust trend
  depiction.
\end{itemize}

\subsection{Software}\label{software}

\textbf{Core data work:} \texttt{tidyverse} (dplyr, tidyr, readr,
ggplot2), \texttt{lubridate}, \texttt{stringr} \textbf{Fast I/O \& large
files:} \texttt{duckdb}, \texttt{DBI}, \texttt{arrow} \textbf{Statistics
\& econometrics:} \texttt{fixest}, \texttt{sandwich}, \texttt{lmtest},
\texttt{clubSandwich}, \texttt{MASS}, \texttt{glmmTMB}, \texttt{pscl}
(optional), \texttt{mgcv}, \texttt{betareg}, \texttt{car},
\texttt{performance}, \texttt{DescTools}, \texttt{Kendall},
\texttt{trend}, \texttt{strucchange}, \texttt{changepoint}
\textbf{Geospatial \& maps:} \texttt{sf}, \texttt{rnaturalearth},
\texttt{rnaturalearthdata}, \texttt{tmap} (optional), \texttt{classInt},
\texttt{viridis}, \texttt{spdep}/\texttt{sfdep} \textbf{Reporting \&
reproducibility:} \texttt{rmarkdown}, \texttt{knitr}, \texttt{renv}

\section{Ethics, Risks, Communication \& Privacy Concerns
Addressed}\label{ethics-risks-communication-privacy-concerns-addressed}

\begin{itemize}
\tightlist
\item
  \textbf{Purpose:} academic analysis of drivers/impacts; \textbf{no
  operational guidance}.
\item
  \textbf{Price data:} use aggregated \textbf{wholesale} ranges/indices;
  no procurement details.
\item
  \textbf{Bias \& uncertainty:} seizures are detection-biased;
  population estimates are ranges; clearly label uncertainty.
\item
  \textbf{Community framing:} avoid victim-blaming; emphasize structural
  drivers and solutions (livelihoods, governance, demand reduction,
  sanitary controls on live trade).
\item
  \textbf{Public-health:} discuss live-trade spillover risk as
  \textbf{associational}.
\item
  \textbf{Causality:} present associations; avoid over-claiming.
\item
  \textbf{Ex-situ data:} document permissions; omit if unavailable.
\item
  \textbf{Attribution:} cite TRAFFIC, CITES, IUCN, UNODC, WJC, and
  others; respect licenses and access terms.
\end{itemize}

\section{Deliverables}\label{deliverables}

\subsection{Visualizations}\label{visualizations}

\subsubsection{F1 --- Tiger-part seizures over time
(total)}\label{f1-tiger-part-seizures-over-time-total}

\textbf{Purpose}: Show overall pressure trend.

\textbf{Inputs (TRAFFIC → Seizures)}: country, year, seizures\_total.
(Optionally sum to global/regional with region from WDI/WGI.)

\textbf{Chart}: Column (year on x, total on y). Optional 3--5-year
moving average.

Recorded tiger-part seizures fluctuate over time, reflecting shifts in
market dynamics, enforcement effort, and reporting capacity.

\subsubsection{F2 --- Seizures by commodity (skins, bones,
teeth/claws)}\label{f2-seizures-by-commodity-skins-bones-teethclaws}

\textbf{Purpose}: Identify which commodities dominate.

\textbf{Inputs (TRAFFIC)}: year, item splits: skins, bones, teeth\_claws
(or long format item, n).

\textbf{Chart}: Small-multiples columns (one panel per item), aligned
axes if feasible.

Different product types exhibit distinct trajectories, indicating
heterogeneous demand and supply chains.

\subsubsection{F3 --- Seizure hot-spots
map}\label{f3-seizure-hot-spots-map}

\textbf{Purpose}: Where enforcement encounters occur.

\textbf{Inputs (TRAFFIC + WDI/WGI)}: country, last 5-yr sum
seizures\_total; join ISO3 to map; optional rate = per million
(seizures\_per\_million).

\textbf{Chart}: Choropleth (country fill by value). Optional labels for
top 5.

Seizures cluster geographically, highlighting priority corridors and
jurisdictions for coordinated responses.

\subsubsection{F4 --- Lagged wholesale index vs.~seizures (``price'' →
pressure
association)}\label{f4-lagged-wholesale-index-vs.-seizures-price-pressure-association}

\textbf{Purpose}: Test if prior wholesale signals align with subsequent
pressure.

\textbf{Inputs (UNODC/WJC + TRAFFIC + WDI/WGI)}: region, year,
index\_mid\_lag; regional seizures\_per\_million.

\textbf{Chart}: Scatter with LOESS (x = index\_mid\_lag, y = regional
seizures\_per\_million).

Higher prior-year wholesale indices are associated with higher recorded
pressure; results are correlational, not causal.

\subsubsection{F5 --- Wild vs.~ex-situ populations
(context)}\label{f5-wild-vs.-ex-situ-populations-context}

\textbf{Purpose}: Contrast conservation states.

\textbf{Inputs (IUCN + ZIMS/studbooks)}:
wild\_pop\_min\textbar max\textbar mid, estimate\_year; ex\_situ\_count,
ex\_situ\_year, coverage\_notes.

\textbf{Chart}: Two bars/lines with error whiskers (wild range
vs.~ex-situ snapshot). Prominent caveat footnote.

Wild populations remain limited relative to ex-situ holdings; estimates
are ranges with substantial uncertainty.

\subsubsection{F6 --- Extinction / extirpation timeline (historic +
recent)}\label{f6-extinction-extirpation-timeline-historic-recent}

\textbf{Purpose}: Place losses in time.

\textbf{Inputs (IUCN / literature)}: event\_type (extinct \textbar{}
functional\_extirpation), event\_year, notes.

\textbf{Chart}: Timeline with labeled markers; panels for (historic
subspecies) vs (recent country losses).

Historic subspecies losses and recent country-level extirpations reflect
interacting drivers, including illegal killing, habitat conversion, and
prey depletion.

\subsubsection{F7 --- Pressure vs.~loss
overlay}\label{f7-pressure-vs.-loss-overlay}

\textbf{Purpose}: Visual association between elevated pressure and loss
events.

\textbf{Inputs (TRAFFIC + events)}: Regional/yearly seizures\_total and
extirpation markers.

\textbf{Chart}: Line (seizures) with vertical lines/flags at
event\_year.

Extirpation milestones coincide with periods of elevated recorded
pressure in parts of the range (descriptive, not causal).

\subsubsection{F8 --- Country case cards (e.g., Cambodia,
Laos)}\label{f8-country-case-cards-e.g.-cambodia-laos}

\textbf{Purpose}: Succinct narrative for key cases.

\textbf{Inputs}: country, last camera-trap year (narrative),
event\_year, 5--10-yr regional seizures\_total sparkline; 2--3 policy
milestones.

\textbf{Chart}: Two mini cards with sparkline and bullets.

Case studies illustrate how poaching and governance conditions can
culminate in functional extirpation.

\subsubsection{F9 --- Poverty vs.~recorded
pressure}\label{f9-poverty-vs.-recorded-pressure}

\textbf{Purpose}: Socioeconomic association.

\textbf{Inputs (WDI/WGI + TRAFFIC)}: pov\_215, log\_spm, region.

\textbf{Chart}: Scatter (x = poverty headcount \%, y = log(1+SPM)),
color by region, LOESS.

Higher poverty aligns with greater recorded pressure; association only,
and subject to detection bias.

\subsubsection{F10 --- Governance vs.~recorded
pressure}\label{f10-governance-vs.-recorded-pressure}

\textbf{Purpose}: Governance association.

\textbf{Inputs (WGI + TRAFFIC)}: rule\_of\_law, log\_spm.

\textbf{Chart}: Scatter with linear fit; optionally facet by region.

Stronger rule of law is associated with lower recorded pressure.

\subsubsection{F11 --- Live-trade risk indicator (spillover
proxy)}\label{f11-live-trade-risk-indicator-spillover-proxy}

\textbf{Purpose}: Track unregulated live-animal trade share.

\textbf{Inputs (TRAFFIC/CITES)}: live\_share by country--year; optional
regional mean.

\textbf{Chart}: Line by region; or bars by country for latest year.

Where live-animal seizures comprise a larger share, spillover risk from
unsanitary holding/transport conditions is of greater concern.

\subsubsection{F12 --- Socioeconomic
quartet}\label{f12-socioeconomic-quartet}

\textbf{Purpose}: Side-by-side drivers view.

\textbf{Inputs (WDI/WGI + TRAFFIC)}: pov\_215, unemp, clean\_water or
electricity\_access, rule\_of\_law, all vs log\_spm.

\textbf{Chart}: 4 small-multiple scatters with identical axes scales.

Poverty, employment, basic services, and governance show distinct
associations with recorded pressure.

\subsubsection{F13 --- Tiger tourism value (site
anchor)}\label{f13-tiger-tourism-value-site-anchor}

\textbf{Purpose}: Ground economic stakes at reserve level.

\textbf{Inputs (Tourism valuation study)}: reserve, anchor\_value\_usd,
gate\_fees\_usd, reserve\_budget\_usd, local\_share.

\textbf{Chart}: Single ``value card'' bar; inset comparing gate fees
vs.~budget.

Tiger tourism at key reserves generates substantial revenue, often
helping finance protection and local livelihoods.

\subsubsection{F14 --- ``What if sightings fall?'' scenario (revenue \&
local earnings
losses)}\label{f14-what-if-sightings-fall-scenario-revenue-local-earnings-losses}

\textbf{Purpose}: Illustrate sensitivity to poaching shocks.

\textbf{Inputs (Tourism + scenario)}: anchor\_value\_usd, local\_share,
visitation\_decline\_\{10,20,30\}, computed revenue\_loss,
local\_earnings\_loss.

\textbf{Chart}: Tornado/interval bar showing loss under 10--30\%
declines; annotate assumptions.

Reduced sightings/access can quickly erode reserve revenue and local
earnings; values shown are illustrative scenarios.

\subsubsection{F15 --- Wildlife-tourism jobs context
(macro)}\label{f15-wildlife-tourism-jobs-context-macro}

\textbf{Purpose}: Put site-level numbers in labor-market perspective.

\textbf{Inputs (WTTC/credible synthesis)}: global and Asia--Pacific
wildlife-tourism jobs (headline stats).

\textbf{Chart}: Two bars or a small multiple comparing totals.

Wildlife tourism supports millions of jobs; declines in flagship species
jeopardize broader employment and development gains.

\subsubsection{F16 --- Reputation \& governance
(callout)}\label{f16-reputation-governance-callout}

\textbf{Purpose}: Communicate non-market, strategic costs.

\textbf{Inputs (UNODC / synthesis)}: short bullets on governance harms,
organized crime links, brand risk.

\textbf{Chart}: Text-forward callout; optional icons; no data plot
required.

Persistent poaching undermines governance and international reputation,
with knock-on effects for investment and destination branding.

\textbf{Notes on joins and readiness per figure}:

\begin{itemize}
\item
  By-country figures: F1--F3, F8--F12 use country--year (ISO3), with
  region as a cosmetic or grouping field.
\item
  By-region figures (price linkage): F4 requires country → region
  mapping to join index\_mid\_lag. Aggregate seizures to region-year
  first.
\item
  Snapshot/context figures: F5 (populations) and F13--F14 (tourism
  anchors) are not merged into the panel; keep as separate tables with
  clear years and caveats.
\item
  Event overlays: F6--F7 add an events table (event\_year, event\_type,
  geography) to annotate timelines.
\end{itemize}

\subsection{Models}\label{models}

\begin{itemize}
\tightlist
\item
  \textbf{Panel OLS (FE):}
  \texttt{fixest::feols(log\_spm\ \textasciitilde{}\ lag(pov\_215,1)\ +\ lag(rule\_of\_law,1)\ +\ lag(gdp\_ppp\_pc,1)\ \textbar{}\ year,\ vcov="HC3")}
\item
  \textbf{Counts (NB):}
  \texttt{MASS::glm.nb(seizures\_total\ \textasciitilde{}\ lag(pov\_215,1)+lag(rule\_of\_law,1)+lag(gdp\_ppp\_pc,1)+factor(year)\ +\ offset(log(pop)))}
\item
  \textbf{Beta regression (live\_share):}
  \texttt{betareg::betareg(live\_share01\ \textasciitilde{}\ clean\_water\ +\ rule\_of\_law\ +\ factor(year))}
\item
  \textbf{Regional price model:}
  \texttt{feols(seizures\_pm\ \textasciitilde{}\ index\_lag\ \textbar{}\ year,\ vcov="NW")}
\end{itemize}

\subsection{Reports}\label{reports}

\begin{itemize}
\tightlist
\item
  \textbf{Format:} This R Markdown renders to \textbf{GitHub Markdown},
  \textbf{PDF}, and \textbf{HTML} with a single source file (this
  \texttt{.Rmd}).
\item
  \textbf{Reproducibility:} Pin package versions with \texttt{renv}.
  Include session info in the appendix.
\item
  \textbf{Artifacts:} Save \texttt{output/figs/} and an analysis-ready
  panel CSV in \texttt{output/data/}.
\end{itemize}

\section{Significance \& Expected
Outcomes}\label{significance-expected-outcomes}

\begin{itemize}
\tightlist
\item
  A clean, analysis-ready \textbf{country--year panel} combining
  seizures, price indices (wholesale ranges), socioeconomic/governance
  indicators, and live-trade proxies.
\item
  Transparent \textbf{figures (F1--F16)} suitable for a policy-facing
  brief.
\item
  Clear statements of \textbf{associations} (not causal claims), with
  ethics and uncertainty front-and-center.
\item
  Practical \textbf{scenario analysis} quantifying potential ecotourism
  revenue and local-earnings losses from reduced sightings/access.
\end{itemize}

\section{Conclusion}\label{conclusion}

Illegal tiger trade is not only a biodiversity crisis; it is a
governance, development, and public-health challenge. By integrating
enforcement signals, market context, socioeconomic conditions, and
tourism economics, this project provides a reproducible evidence base
for interventions that pair \textbf{livelihoods and governance} with
\textbf{enforcement and demand reduction}.

\section{Appendices}\label{appendices}

\subsection{A. Package/session info}\label{a.-packagesession-info}

\begin{Shaded}
\begin{Highlighting}[]
\FunctionTok{sessionInfo}\NormalTok{()}
\end{Highlighting}
\end{Shaded}

\begin{verbatim}
## R version 4.2.2 Patched (2022-11-10 r83330)
## Platform: x86_64-pc-linux-gnu (64-bit)
## Running under: Debian GNU/Linux 12 (bookworm)
## 
## Matrix products: default
## BLAS:   /usr/lib/x86_64-linux-gnu/blas/libblas.so.3.11.0
## LAPACK: /usr/lib/x86_64-linux-gnu/lapack/liblapack.so.3.11.0
## 
## locale:
##  [1] LC_CTYPE=en_US.UTF-8       LC_NUMERIC=C              
##  [3] LC_TIME=en_US.UTF-8        LC_COLLATE=en_US.UTF-8    
##  [5] LC_MONETARY=en_US.UTF-8    LC_MESSAGES=en_US.UTF-8   
##  [7] LC_PAPER=en_US.UTF-8       LC_NAME=C                 
##  [9] LC_ADDRESS=C               LC_TELEPHONE=C            
## [11] LC_MEASUREMENT=en_US.UTF-8 LC_IDENTIFICATION=C       
## 
## attached base packages:
## [1] stats     graphics  grDevices utils     datasets  methods   base     
## 
## loaded via a namespace (and not attached):
##  [1] compiler_4.2.2    fastmap_1.2.0     cli_3.6.5         tools_4.2.2      
##  [5] htmltools_0.5.8.1 yaml_2.3.10       rmarkdown_2.29    knitr_1.50       
##  [9] xfun_0.53         digest_0.6.37     rlang_1.1.6       evaluate_1.0.5
\end{verbatim}

\subsection{B. Hot-spot map scaffold}\label{b.-hot-spot-map-scaffold}

\begin{Shaded}
\begin{Highlighting}[]
\FunctionTok{library}\NormalTok{(sf); }\FunctionTok{library}\NormalTok{(rnaturalearth); }\FunctionTok{library}\NormalTok{(ggplot2); }\FunctionTok{library}\NormalTok{(viridis)}
\NormalTok{world }\OtherTok{\textless{}{-}}\NormalTok{ rnaturalearth}\SpecialCharTok{::}\FunctionTok{ne\_countries}\NormalTok{(}\AttributeTok{scale =} \StringTok{"medium"}\NormalTok{, }\AttributeTok{returnclass =} \StringTok{"sf"}\NormalTok{) }\SpecialCharTok{|\textgreater{}}
\NormalTok{  dplyr}\SpecialCharTok{::}\FunctionTok{select}\NormalTok{(iso\_a3, geometry)}

\CommentTok{\# last5y\_panel must contain: country\_iso3, seizures\_per\_million}
\NormalTok{mapdat }\OtherTok{\textless{}{-}}\NormalTok{ world }\SpecialCharTok{|\textgreater{}}\NormalTok{ dplyr}\SpecialCharTok{::}\FunctionTok{left\_join}\NormalTok{(last5y\_panel, }\AttributeTok{by =} \FunctionTok{c}\NormalTok{(}\StringTok{"iso\_a3"} \OtherTok{=} \StringTok{"country\_iso3"}\NormalTok{))}

\FunctionTok{ggplot}\NormalTok{(mapdat) }\SpecialCharTok{+}
  \FunctionTok{geom\_sf}\NormalTok{(}\FunctionTok{aes}\NormalTok{(}\AttributeTok{fill =}\NormalTok{ seizures\_per\_million), }\AttributeTok{color =} \ConstantTok{NA}\NormalTok{) }\SpecialCharTok{+}
  \FunctionTok{scale\_fill\_viridis\_c}\NormalTok{(}\AttributeTok{option =} \StringTok{"C"}\NormalTok{) }\SpecialCharTok{+}
  \FunctionTok{labs}\NormalTok{(}\AttributeTok{fill =} \StringTok{"Seizures / 1M"}\NormalTok{, }\AttributeTok{title =} \StringTok{"Tiger{-}part seizure hot spots (last 5 years)"}\NormalTok{) }\SpecialCharTok{+}
  \FunctionTok{theme\_minimal}\NormalTok{()}
\end{Highlighting}
\end{Shaded}

\subsection{C. Lagged wholesale index vs.~seizures
(regional)}\label{c.-lagged-wholesale-index-vs.-seizures-regional}

\begin{Shaded}
\begin{Highlighting}[]
\FunctionTok{library}\NormalTok{(dplyr); }\FunctionTok{library}\NormalTok{(ggplot2)}
\NormalTok{regdat }\OtherTok{\textless{}{-}}\NormalTok{ region\_year }\SpecialCharTok{|\textgreater{}}
  \FunctionTok{mutate}\NormalTok{(}\AttributeTok{seizures\_pm =} \FloatTok{1e6} \SpecialCharTok{*}\NormalTok{ seizures\_total }\SpecialCharTok{/}\NormalTok{ pop,}
         \AttributeTok{index\_lag   =}\NormalTok{ dplyr}\SpecialCharTok{::}\FunctionTok{lag}\NormalTok{(index\_mid, }\DecValTok{1}\NormalTok{)) }\SpecialCharTok{|\textgreater{}}
\NormalTok{  tidyr}\SpecialCharTok{::}\FunctionTok{drop\_na}\NormalTok{(seizures\_pm, index\_lag)}

\FunctionTok{ggplot}\NormalTok{(regdat, }\FunctionTok{aes}\NormalTok{(index\_lag, seizures\_pm)) }\SpecialCharTok{+}
  \FunctionTok{geom\_point}\NormalTok{(}\AttributeTok{alpha =} \FloatTok{0.7}\NormalTok{) }\SpecialCharTok{+}
  \FunctionTok{geom\_smooth}\NormalTok{(}\AttributeTok{method =} \StringTok{"loess"}\NormalTok{, }\AttributeTok{se =} \ConstantTok{FALSE}\NormalTok{, }\AttributeTok{color =} \StringTok{"black"}\NormalTok{) }\SpecialCharTok{+}
  \FunctionTok{labs}\NormalTok{(}\AttributeTok{x =} \StringTok{"Wholesale index (lagged 1y)"}\NormalTok{, }\AttributeTok{y =} \StringTok{"Seizures per million"}\NormalTok{,}
       \AttributeTok{title =} \StringTok{"Wholesale signal (lagged) vs. recorded pressure"}\NormalTok{) }\SpecialCharTok{+}
  \FunctionTok{theme\_minimal}\NormalTok{()}
\end{Highlighting}
\end{Shaded}


\end{document}
